\documentclass{article}
\usepackage[utf8]{inputenc}
\usepackage[T1]{fontenc}
\usepackage[nil, spanish]{babel}
\usepackage{graphicx, tikz, pgfplots}
\usepackage{amsmath, amsthm, amssymb}
\usetikzlibrary{positioning}
\newtheorem*{theorem}{Teorema}
\newtheorem*{definition}{Definición}

\title{Análisis Matemático I}
\date{}
\author{}

\begin{document}
  \maketitle
  \section*{Límites:} 
    \begin{definition}[límite formal]
      \[
        \lim_{x\to c}f(x)=L\iff\forall\epsilon>0 \ \exists\ \delta>0\ :0<|x-c|<\delta\Longrightarrow
        |f(x)-L| 
      \]
    \end{definition}
    \begin{definition}[límite intuitivo]
      Decir que $\lim_{x\to c}f(x)=L$ significa  que cuando $x$ está cerca pero diferente de $c$,
      entonces $f(x)$ está cerca de $L$.
    \end{definition}
    \begin{theorem}[unicidad del límite]
      Si el límite de una función existe, entonces es único.
    \end{theorem}
    \begin{proof}
      Supongamos que $\lim_{x\to c}f(x)=L$ y  $\lim_{x\to c}f(x)=L'$ siendo $L$ y $L'$
    \end{proof}
    \begin{theorem}[del emparedado]
      Sean $f$, $g$ y $h$ funciones que satisfacen $f(x)\leqslant g(x)\leqslant h(x)\forall x$
      cercano a $c$, excepto posiblemente $c$. Si $\lim_{x\to c}f(x)=\lim_{x\to c}h(x)=L$, entonces
      $\lim_{x\to c}g(x)=L$.
    \end{theorem}
    \begin{proof}
      Sea $\epsilon>0$. Elegimos $\delta_1$ tal que
      \[
        0<|x-c|<\delta_1\Longrightarrow L-\epsilon<f(x)<L+\epsilon
      \]
      y $\delta_2$ tal que
      \[
        0<|x-c|<\delta_2\Longrightarrow L-\epsilon<h(x)<L+\epsilon
      \]
      Elegimos $\delta_3$ de modo que
      \[
        0<|x-c|<\delta_3\Longrightarrow f(x)\leqslant g(x)\leqslant h(x)
      \]
      Sea $\delta=$mín\{$\delta_1,\delta_2,\delta_3$\}. Entonces
      \[
        0<|x-c|<\delta\Longrightarrow L-\epsilon<f(x)\leqslant g(x)\leqslant h(x)<L+\epsilon
      \]
      Concluímos que $\lim_{x\to c}g(x)=L$
    \end{proof}
    \newpage

  \section*{Continuidad:}
    \begin{definition}[continuidad en un punto]
      Sea $f$ definida en un intervalo abierto que contiene a $c$. Decimos que $f$ es 
      continua en $c$ si
      \[
        \lim_{x\to c}f(x)=f(c)
      \]
    \end{definition}
    \begin{theorem}[Bolzano]
      Sea $f$ una función continua y definida en $[a,b]$. Si se cumple que $f(a)<0<f(b)$ o 
      $f(b)<0<f(a)$, entonces existe al menos un punto $c\in(a,b)$ tal que $f(c)=0$.
    \end{theorem}
    \begin{proof}
      Sea $f$ una función continua y definida en $[a,b]$ y $f(a)<0<f(b)$. Sea $C_+$ un
      conjunto tal que
      \[
        C_+=\{x\in[a,b]/f(x)\geqslant0\}
      \]
      Sea $c\in[a,b]$ el supremo del conjuto $C_+$, entonces $\exists[c-\delta,c+\delta]=
      signo\ de\ f(c)$ (por teorema de la conservación del signo). Si suponemos que $f(c)<0$
    \end{proof}
    
    \begin{theorem}[valor intermedio]
      Sea $f$ una función continua y definida en $[a,b]$ y $k\in(a,b)$ tal que $f(a)<k<f(b)$,
      entonces existe $c\in(a,b)$ tal que $f(c)=k$.
    \end{theorem}
    \begin{theorem}[máximos y mínimos]
    \end{theorem}
    
\end{document}
